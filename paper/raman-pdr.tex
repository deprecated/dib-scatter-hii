\RequirePackage{amsmath}
\documentclass[twocolumn, times]{aastex63}
\usepackage[spanish,es-minimal,english]{babel}
\usepackage[utf8]{inputenc}
\usepackage{natbib}
%\usepackage{microtype}
\usepackage{hyperref}
\usepackage{savesym}
\savesymbol{tablenum}
\usepackage{siunitx}
\restoresymbol{SIX}{tablenum}
\usepackage[varg]{newtxmath}
\usepackage{newtxtext}



\bibliographystyle{aasjournal}

\newcommand\ION[2]{#1\,\scalebox{0.9}[0.8]{\uppercase{#2}}}
\newcounter{ionstage}
\renewcommand{\ion}[2]{\setcounter{ionstage}{#2}% 
  \ensuremath{\mathrm{#1\,\scriptstyle\Roman{ionstage}}}}
\newcommand\hii{\ion{H}{2}}
\newcommand\Raman{\ensuremath{_{\text{Raman}}}}
\def\th#1#2{\(\theta^{#1}\)\,Ori~#2}


\begin{document}
\title{Raman mapping of atomic hydrogen in the Orion Bar and Orion South}
\shorttitle{Raman mapping of atomic hydrogen in Orion}
\author{William J. Henney}
\affiliation{%
  \foreignlanguage{spanish}{Instituto de Radioastronomía y
    Astrofísica, Universidad Nacional Autónoma de México, Apartado
    Postal 3-72, 58090 Morelia, Michaoacán, Mexico}}
\email{w.henney@irya.unam.mx}

\begin{abstract}
  I show that the broad Raman-scattered wings of H\(\alpha\) can be used to
  map neutral gas illuminated by high-mass stars in star forming
  regions. The near wings (\(\Delta\lambda \approx \pm \SI{10}{\angstrom}\)) trace neutral columns Absorption features in the pseudo-continuum at 6634 and
  6663~\AA{} correspond to neutral oxygen far-ultraviolet absorption
  lines at \SIlist{1027.43;1028.16}{\angstrom}.
\end{abstract}

\keywords{Atomic physics; Radiative transfer; Photodissociation regions}
\facilities{VLT:Yepun (MUSE); OANSPM:2.1m (Mezcal)}
\object{M42}
\section{Introduction}
\label{sec:introduction}

\citet{Dopita:2016a} were the first to identify Raman scattering in the Orion Nebula.

\citet{Dopita:2016a} propose that the Raman wings form at the
transition zone near the ionization fronts in \hii{} regions.
However, the total neutral hydrogen column through the ionization
front can be no more than about \(10 / \sigma_0\), where
\(\sigma_0 \approx \SI{6.3e-18}{cm^2}\).  The Raman scattering cross section at
wavelengths responsible for the observed wings is much lower than
this: \(\sigma\Raman \sim \SI{1e-22}{cm^2}\) \citep{Chang:2015a}, meaning that
the Raman scattering optical depth through the ionization front is
only of order \(0.0001\).  A vastly larger column density of neutral
hydrogen is available in the photodissociation region outside the
ionization front, so it is more likely that Raman scattering will
occur there instead, so long as there is sufficient far ultraviolet
radiative flux in the vicinity of the Lyman~\(\beta\) line
(\SI{1025}{\angstrom}).

\section{Observations}
\label{sec:observations}

MUSE \citep{Bacon:2010a} observations of the Orion Nebula \citep{Weilbacher:2015a, Mc-Leod:2015b}.

\section{Discussion}
\label{sec:discussion}

The effective resolving power of the optical spectrograph is multiplied by 6.4 for the FUV domain.

The \ion{O}{1} lines should be in absorption in the spectrum seen by the Raman scatterers. 

\citet{Salgado:2016a} had found low dust cross-section in Orion Bar
PDR, but there are loopholes. First, they assume plane-parallel
geometry with exactly edge-on viewing angle, while in reality it is a
roughly cylindrical filament.  Second, they ignore scattering, see
\citet{Watson:1998a}.

\bibliography{BibdeskLibrary}


\end{document}


\end{document}
%%% Local Variables:
%%% mode: latex
%%% TeX-master: t
%%% End:
